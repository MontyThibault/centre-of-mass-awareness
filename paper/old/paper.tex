
\documentclass[]{article}

\usepackage{amssymb}
\usepackage{amsmath}


\title{Uncertainty in Centre of Mass and Transformation Frames Applied to Physically-Based Character Animation Systems}
\author{Monty Thibault, Paul Kry}

\begin{document}

\maketitle

\begin{abstract}
We present a 
\end{abstract}

\section{Related Work}
\subsection{Physically-Based Animation Systems}

Traditional character animation systems used in film \& games rely artist-driven keyframing techniques. While proven to be effective, keyframing is known to be time-consuming and highly dependent on the skill of the artist.

\subsection{Human Sensory Awareness}

Significant study has been done in the field of neuroscience on the brain's representation of the human body. This topic has been coined various names self-attribution \cite{2010-TOG-gbwc}, the brain-body model \cite{}, and ... . However, there has not been a call to quantify this work into representations that can recreate convincing human motor error; integration into physically-based character animation systems may be a unique use-case in this instance.

\section{A Model of COM Estimation}

We use a function $g: \mathbb{R}^3 \rightarrow \mathbb{R}$ to model the relative certainty that the centre of mass, $C_M$ lies at point $p \in \mathbb{R}^3$. The function itself is composed of the following components.

\begin{gather}
g = \alpha g_0 + V + \sum{J_R + J_V}
\end{gather}

Where $g_0$ is the distribution of the previous frame, $V$ is the velocity of the previous frame, $\alpha$ is a damping constraint, $J_R$ is the change from joint rotations, and $J_V$ is the change from joint torques.


We then consider a stochastic walk of distance $d$ from the last estimate of $C_M$ weighted along the gradient of the function $g$. $d$ is given precisely as the change in the function $g$ monte-carlo sampled about the point $C_M$.

\begin{gather}
d = \iiint_M (g_0 - g)dV
\end{gather}


\section{}
TODO

\begin{enumerate}
	\item{Better random sampling function. (more random sins?)}
	\item{g function setup.}
	\item{Volume integral setup.}
\end{enumerate}

\bibliographystyle{plain}
\bibliography{paper}

\end{document}
